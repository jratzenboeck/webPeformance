\chapter*{Abstract}
\markright{Abstract}

There have been dramatic changes over the last years regarding the web, it is content, purpose and complexity. Websites are not just simple HTML (Hyper Text Markup Language) files anymore which reside on a server at a single location. Many businesses operate highly available web applications which often provide a lot of functionality delivered anytime to users around the world. Customers expect those websites to be fast, otherwise they just leave. Several studies showed that good web performance is a "must-have" feature in modern and beloved web applications and increases chances of attracting potential customers evidently. 

The requirement of high performance adds another activity to the multi-disciplinary field of web development. This work explains the important factors latency and bandwidth dictating the time messages spend on the way between client and destination server. Some examples outline the high influence of the distance between sender and receiver, how Latency and Bandwidth correlate and where bottlenecks possibly occur. Following that the browser has to process incoming responses and build up the web pages as the user finally perceives it which introduces further delays. The correct and intelligent usage of HTML, CSS (Cascading Style Sheet), JavaScript and multimedia content affects user experience positively. 

For the development of the new website of the pet tracking company Tractive the current system was monitored with several tools to find out potential bottlenecks. Different load tests gave some indication of how the servers and the back-end performs under stress while browser processing investigations were leveraged to detect blocking and performance critical resources. Finally some suggestions to tackle existing problems are presented which give an idea of how some things can be improved or re-implemented to support high performance.