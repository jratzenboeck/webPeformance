\chapter{Introduction}
\label{sec:introduction}
In 1990 the first website was published on the Internet. Although designed for world wide availability, the actual number of users who  had access to the Internet was rather limited at these times. Over the years the Web has evolved dramatically with respect to content volume, media diversity and user population. Nowadays  the internet is nearly inevitable for successful world wide businesses. Companies present themselves with websites, sell their products and provide services for their customers online. 

A good web experience is therefore crucial and the speed of a website often plays a key role whether an interested visitor becomes a customer or not~\cite{menasce2000scaling}. 
Therefore it is a factor which a successful company should not neglect when developing and maintaining a website. This thesis documents the detailed investigation of the performance of a business website and outlines possibilities to improve the perceived web experience based on the Upper Austrian pet tracking company Tractive GmbH.  

\section{Motivation}

Throughout all the innovative developments in telecommunication the web browser might have become the most widespread deployment platform to developers. Industry growth projections~\cite{Grigorik_2013} predict an amount of 20 billion connected devices by 2020 and each device comes with a browser installed. In contrast to all different applications which are somehow tailored to fulfil some defined tasks for a given targeted audience, the browser does not have any limitations disregarded of the type of the device, used platform or manufacturer. The browser from the early days looks nearly nothing like the one we use nowadays on our devices. Concurrently with these technical innovations the variety and complexity of tasks which can be done via the Web has increased as well and the user expectations on web applications are very high. This is why web performance definitely has to be considered a feature next to design for instance. Empirical studies showed that users tend to drop away from slow websites whereas faster websites lead to
	\begin{itemize}
		\item better user engagement
		\item better user retention
		\item higher conversion rates (indicates how many of the website visitors convert the paying customers)
	\end{itemize}
Single page applications, responsive designs, cool CSS3 animations or a bunch of new Web User Interface components featured by HTML5 are only a few things which come along with modern web applications as they appear in our today's web browsers. All these innovations add complexity and therefore more data to be handled throughout the network. This introduces a further challenge for web engineers as all this design tweaks require more work to deliver high performance web applications to their users.    
Understanding how the web with it is different underlying protocols works is essential. Good developers have to have some deeper knowledge of how the bits within a HTTP (Hyper Text Transfer Protocol) request are transported from one point to another as well as how the browser on the client side renders HTML, loads media and processes JavaScript~\cite{Grigorik_2013}.


\section{What is Tractive?}

Tractive GmbH produces hardware devices for tracking pets and develops software applications which allow users to communicate with their pets wearing these devices. The company was founded in Pasching, Upper Austria in 2012 and offers today several different devices whereby the origin goal was to find and locate pets fitted with a GPS (Global Positioning System) tracker using a Smartphone App. In addition to the hardware devices and mobile apps Tractive develops and maintains a website providing all the necessary product information for potential customers as well as a web shop which allows customers to purchase products and a web application where registered users can manage their trackers and pets. 

The company is interested in analysing and improving the performance of their website, including all de-facto static content as well as the mentioned Pet Manager. Relevant components are to be re-designed as well as integrated and connected to the current backend containing the data and business services used by client applications. Next to a good "look and feel" it is important for the company to evaluate how well the current website performs and try to find bottlenecks which should be considered and fixed when integrating the new website.     

\section{Objective and Outline of the Thesis}

This thesis will contribute towards a deeper understanding of performance influencing factors of modern websites.
It will therefore start with an overview of web performance metrics typically used in performance evaluation studies (see chapter~\ref{chap:metrics}).
To relate those metrics to the actual performance behaviour of a real world website, the Tractive case study will be used. 
In chapter \ref{chap:architecture} the architecture of the system under test will be briefly described.
The next chapter~\ref{chap:measurements} will discuss the  experimental design of the measurements as well as the results.
Based on this, suggested improvements~\ref{chap:improvements}  and an outlook on future work~\ref{chap:conclusion} will conclude the thesis.