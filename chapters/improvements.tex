\chapter{Suggested Improvements}
\label{chap:improvements}

The previous chapter covered the measurements which were taken to determine the overall web performance of the system under test. Furthermore some problems and potential bottlenecks affecting performance were outlined. The following sections try to tackle those problems and suggest possible improvements.

\section{Content Delivery Networks (CDN)}

The measurements regarding time spent in the network confirmed, that as latency is a major issue and one of the most important metrics which drives loading speed of a website, engineers have to care about it. CDNs are a great opportunity to improve network latency by bringing the content closer to the clients. Since Tractive operates in over 80 countries of the world, clients necessarily experience different latencies because the website is mirrored on three hosts residing in one datacenter in Germany. A CDN offers the possibility to replicate the content across a set of geographically distributed servers instead of putting them only at a single location. Based on the latency (which is mostly driven by the distance) the client gets redirected to the nearest node to get the requested content which minimizes round-trip-times for TCP connections, SSL handshakes and so on \cite{Google_2009}. These geographically distributed nodes, which are called edge-servers, cache the requested content after the first time. So if the desired content is already available at the closest edge server it can be directly delivered and no extra round trip to the origin server is necessary. A popular CDN is Amazon Cloud Front which allows to distribute content of the website globally in Amazon's Availability zones. Cloud Front integrates perfectly if the web content is hosted in an Amazon Web Services environment \cite{Amazon_2015}.

CDNs can be used to deliver the entire website but also for single assets like images, JavaScript etc. Within the new website for instance different JavaScript libraries are used. In the web application these are provided via the CloudFare CDN instead of storing them at the origin server in Germany. This should help to improve the critical rendering path and force loading time of the entire website to decrease.

\begin{lstlisting}[caption=Loading JavaScript libraries from Cloudfare CDN]
<script src="//cdnjs.cloudflare.com/ajax/libs/store.js/1.3.17/store.min.js" charset="utf-8"></script>
<script src="//cdnjs.cloudflare.com/ajax/libs/jquery/1.9.1/jquery.min.js" type="text/JavaScript"></script>
<script src="//cdnjs.cloudflare.com/ajax/libs/flexslider/2.1/jquery.flexslider.js" type="text/JavaScript"></script>
<script src="//cdnjs.cloudflare.com/ajax/libs/flickity/0.1.0/flickity.pkgd.min.js" type="text/JavaScript"></script>
\end{lstlisting}

\section{Loading of Assets and Caching}

Images, nice layouts using CSS, fancy animations or jQuery AJAX calls to back-end services, all of them are critical in the rendering path and have to be loaded and executed fast to avoid diminishing web performance. 
As the measurements in ~\ref{sec:browserProcessing} showed the main application.js file is a potential bottleneck because it delays the start of rendering significantly especially in case of high latencies. Fortunately Caching supports an intelligent solution to prevent unnecessary loading of assets from the origin server each time. Especially de-facto static content like product or advertising images on the main website should be cached within the browser to enable good user experience at recurring page visits. Fortunately we have Nginx as powerful reverse proxy and intermediary web cache sitting in between the load balancer and the application server. Nginx offers a rich set of caching policies to avoid requests always forwarded to the application server and asking for a resource. In the current state caching for static assets is leveraged enough but Nginx allows fine-tuning configuration even for dynamic content. Revalidation proxy cache setting for instance allows to check if there were any changes to dynamic resource and even if the cache header has expired delivering cached content. 
 
\section{Compression}

When content has to be downloaded, file size influences visualization performance even though parallel downloads are often no problem. To get fast download times assets should be compressed with GZip which is supported by all popular web server. Also Nginx supports gzip static compression.

\section{Language and Redirection}

Providing web content in different languages is a duty for Tractive to be competitive within the global market. However, this introduces some challenges because the recommended language first has to be detected correctly and the web application has to know which set of translations it should take. To solve this Tractive used a simple redirect to indicate the used language (for instance tractive.com/en) directly after the first request. The measurement results showed that this is problematic for the starting render time since it involves a new HTTP request and further round trips. There are several possible solutions to solve this issue. The new website uses a JavaScript file to detect the browser's language and compares it with the valid languages Tractive supports. As fallback language English is used. The following code snippet shows a part of the lang.js which reduces the time spent in the network from about 140ms to 45ms for a request from Frankfurt, Germany.

\begin{lstlisting}[caption=Code snippet of lang.js]
var validLangs = ["en", "de", "es", "it", "fr", "sr", "ru", "bg", "hr", "cs"];

// Get the browser language
function getBrowserLang() {
    return navigator.language.split('-')[0];
 }
 // Return the browser language if supported, fallback to English otherwise
function getChosenLang() {
    var bl = getBrowserLang();
    return validLangs.indexOf(bl) < 0 ? validLangs[0] : bl;
}
// Get localized root URL
function getLangRoot(lang) {
    return "/" + lang;
}
\end{lstlisting} 

After executing the lang.js file the appropriate HTML file in the correct language can be generated and returned to the client. With a Rails application running in the back-end there is an even better solution to set the correct language. Since Rails generates the outcoming HTML with its template engine ERB which listens to a global variable "locale", we can set this variable in the main application controller based on the Accept Language HTTP header. This allows to remove the JavaScript file from above completely and saves time before constructing the DOM and CSSOM. The only thing the Rails application server has to do is to download the Accept Language Header which is just a few kilobytes. Especially for dynamic websites which need a rich back-end anyway such a solution makes sense whereas static websites such as the new developed main website tractive.com do not use a Rails application behind it anymore. Instead HTML pages are generated for the languages with some middleware software.
